% Options for packages loaded elsewhere
\PassOptionsToPackage{unicode}{hyperref}
\PassOptionsToPackage{hyphens}{url}
%
\documentclass[
  8pt,
  ignorenonframetext,
]{beamer}
\usepackage{pgfpages}
\setbeamertemplate{caption}[numbered]
\setbeamertemplate{caption label separator}{: }
\setbeamercolor{caption name}{fg=normal text.fg}
\beamertemplatenavigationsymbolsempty
% Prevent slide breaks in the middle of a paragraph
\widowpenalties 1 10000
\raggedbottom
\setbeamertemplate{part page}{
  \centering
  \begin{beamercolorbox}[sep=16pt,center]{part title}
    \usebeamerfont{part title}\insertpart\par
  \end{beamercolorbox}
}
\setbeamertemplate{section page}{
  \centering
  \begin{beamercolorbox}[sep=12pt,center]{part title}
    \usebeamerfont{section title}\insertsection\par
  \end{beamercolorbox}
}
\setbeamertemplate{subsection page}{
  \centering
  \begin{beamercolorbox}[sep=8pt,center]{part title}
    \usebeamerfont{subsection title}\insertsubsection\par
  \end{beamercolorbox}
}
\AtBeginPart{
  \frame{\partpage}
}
\AtBeginSection{
  \ifbibliography
  \else
    \frame{\sectionpage}
  \fi
}
\AtBeginSubsection{
  \frame{\subsectionpage}
}
\usepackage{amsmath,amssymb}
\usepackage{lmodern}
\usepackage{iftex}
\ifPDFTeX
  \usepackage[T1]{fontenc}
  \usepackage[utf8]{inputenc}
  \usepackage{textcomp} % provide euro and other symbols
\else % if luatex or xetex
  \usepackage{unicode-math}
  \defaultfontfeatures{Scale=MatchLowercase}
  \defaultfontfeatures[\rmfamily]{Ligatures=TeX,Scale=1}
\fi
\usetheme[]{Malmoe}
\usecolortheme{whale}
\usefonttheme{professionalfonts}
% Use upquote if available, for straight quotes in verbatim environments
\IfFileExists{upquote.sty}{\usepackage{upquote}}{}
\IfFileExists{microtype.sty}{% use microtype if available
  \usepackage[]{microtype}
  \UseMicrotypeSet[protrusion]{basicmath} % disable protrusion for tt fonts
}{}
\makeatletter
\@ifundefined{KOMAClassName}{% if non-KOMA class
  \IfFileExists{parskip.sty}{%
    \usepackage{parskip}
  }{% else
    \setlength{\parindent}{0pt}
    \setlength{\parskip}{6pt plus 2pt minus 1pt}}
}{% if KOMA class
  \KOMAoptions{parskip=half}}
\makeatother
\usepackage{xcolor}
\newif\ifbibliography
\setlength{\emergencystretch}{3em} % prevent overfull lines
\providecommand{\tightlist}{%
  \setlength{\itemsep}{0pt}\setlength{\parskip}{0pt}}
\setcounter{secnumdepth}{-\maxdimen} % remove section numbering
\usepackage{booktabs}
\ifLuaTeX
  \usepackage{selnolig}  % disable illegal ligatures
\fi
\IfFileExists{bookmark.sty}{\usepackage{bookmark}}{\usepackage{hyperref}}
\IfFileExists{xurl.sty}{\usepackage{xurl}}{} % add URL line breaks if available
\urlstyle{same} % disable monospaced font for URLs
\hypersetup{
  pdftitle={Homofilia y desigualdad en Chile},
  pdfauthor={Cantillan, R.; Espinoza,V.; Bargsted, M. \& Plaza, A.},
  hidelinks,
  pdfcreator={LaTeX via pandoc}}

\title{Homofilia y desigualdad en Chile}
\subtitle{Seminario ELSOC-COES 2022}
\author{Cantillan, R.; Espinoza,V.; Bargsted, M. \& Plaza, A.}
\date{junio 21, 2022}
\institute{ISUC PUC - COES - redesLab}

\begin{document}
\frame{\titlepage}

\begin{frame}[allowframebreaks]
  \tableofcontents[hideallsubsections]
\end{frame}
\begin{frame}{Motivación}
\protect\hypertarget{motivaciuxf3n}{}
\begin{itemize}
\item
  En Chile, las reformas neoliberales establecidas durante la década de
  1980, se tradujeron en un proyecto de modernización orientado al
  desarrollo de

  \begin{itemize}
  \tightlist
  \item
    Estabilidad política
  \item
    Crecimiento económico sostenido (pro pobre)
  \end{itemize}
\item
  Una de las consecuencias perniciosas de este proyecto político es el
  crecimiento sostenido de la desigualdad socioeconómica y los
  consecuentes problemas de cohesión social.
\end{itemize}
\end{frame}

\begin{frame}{Motivación II}
\protect\hypertarget{motivaciuxf3n-ii}{}
\begin{itemize}
\tightlist
\item
\end{itemize}
\end{frame}

\begin{frame}{La homofilia está en todas partes!}
\protect\hypertarget{la-homofilia-estuxe1-en-todas-partes}{}
\begin{itemize}
\item
  La \textbf{homofilia} es un fenómeno ecólogico y patrón empírico de
  organización de las sociedades.
\item
  Refiere a la tendencia empírica en la cual el contacto entre personas
  similares ocurre a un ritmo mayor que entre personas diferentes.
\item
  La omnipresencia de la homofilia significa que la información
  cultural, conductual, genética o material que fluye a través de las
  redes tenderá a estar localizada .
\item
  Esta tendencia constituye una de las características más marcadas de
  la estructura social de las sociedades modernas (McPherson, 2009;
  Smith et al., 2014).
\end{itemize}
\end{frame}

\begin{frame}{Relevancia}
\protect\hypertarget{relevancia}{}
\begin{itemize}
\item
  Realidades opuestas no se atraen
\item
  Potenciar homofilia como mecanismo.
\item
  Relevancia de lo político. (relaciones no monotónicas)
\item
\item
  Homofilia como correlato micro de la desigualdad social.
\item
  Homofilia multidimensional (segregación).
\item
  Caso control (técnica de punta)
\end{itemize}
\end{frame}

\begin{frame}{Objetivo}
\protect\hypertarget{objetivo}{}
\begin{itemize}
\item
  En este artículo nos preguntamos hasta qué punto la modernización ha
  desdibujado las barreras entre los grupos sociales en Chile.
\item
  El patrón chileno en las relaciones sociales mostró límites bien
  establecidos entre grupos sociales como personas educadas y menos
  escolarizadas, mujeres y hombres, católicos y no católicos (Bargsted
  Valdés et al., 2020).
\item
  Fronteras eran menos claras, entre los jóvenes y las personas mayores
  e incluso entre las personas de inclinaciones políticas divergentes.
\item
  La expansión del sistema educativo, especialmente el acceso a la
  educación universitaria, ha debilitado las barreras en muchas de las
  dimensiones, aunque el nivel socioeconómico sigue estructurando la
  sociedad chilena.
\item
  Argumentamos, primero que los patrones en la interacción social son
  consistentes con lo esperado bajo un proceso de modernización y,
  sostenemos que estos rasgos se mantienen estables durante el período
  en estudio porque expresan la estructura social de la sociedad
  chilena.
\end{itemize}
\end{frame}

\begin{frame}{Teoría I}
\protect\hypertarget{teoruxeda-i}{}
\begin{itemize}
\item
  El análisis clásico de Lazarsfeld y Merton (1954) sobre las
  preferencias de adultos en la selección de amistades, usó el término
  homofilia para referirse a \textbf{``una tendencia en la formación de
  amistades entre aquellos que son similares en algún aspecto
  designado''}, ya sea, de carácter innato o adquirido.
\item
  La \textbf{homofilia}, entonces, se refiere a la mayor probabilidad de
  interacción entre personas de características similares (McPherson et
  al., 2001). La asociación entre dos o más personas va frecuentemente
  acompañada de similitudes de educación, ocupación, área de residencia,
  estilo de vida, opiniones o creencias religiosas.
\item
  En efecto, relaciones sociales institucionalizadas, tales como el
  matrimonio o la amistad tienden a ocurrir con mayor probabilidad entre
  personas similares, lo cual demarca un aspecto que no es parte de la
  normatividad asociada con estas relaciones.
\end{itemize}
\end{frame}

\begin{frame}{Teoría II}
\protect\hypertarget{teoruxeda-ii}{}
\begin{itemize}
\item
  Homofilia = alta integración local y baja integración global.
\item
  La \textbf{integración local} se caracteriza por la presencia de
  grupos de alta cohesión, con lazos fuertes entre sus miembros, que
  facilitan la coordinación necesaria para la acción colectiva o la
  circulación de recursos (en este sentido, la homofilia puede ser
  considerada como un \textbf{mecanismo de clausura}).
\item
  La \textbf{integración global}, por otro lado, se refiere a la
  vinculación entre círculos sociales diversos a través de vínculos
  débiles, acompañada por dificultades de coordinación y establecimiento
  de reglas compartidas.
\item
  La consistencia interna de un grupo, llevada al extremo, puede
  convertirse en una desventaja asociada con la marginalización. En
  contraste, las vinculaciones externas favorecen la cohesión social
  global.
\end{itemize}
\end{frame}

\begin{frame}{Teoría III (convergencia Blau/ Granovetter)}
\protect\hypertarget{teoruxeda-iii-convergencia-blau-granovetter}{}
\begin{itemize}
\item
  \textbf{Interacciones macro}: vinculaciones entre posiciones sociales
  Blau (1974, 1977).
\item
  Las sociedades poseen un grado de \textbf{heterogeneidad} que depende
  del número de posiciones sociales identificables. Las sociedades de
  alta heterogeneidad -compuestas por muchas posiciones sociales de
  similar tamaño- ofrecen mayores oportunidades de interacción entre
  integrantes de distintos grupos que las sociedades más homogéneas
  (Blau, 1977). En sociedades heterogéneas, la distribución de la
  población en posiciones sociales se correlaciona débilmente entre sí.
\item
  Las sociedades \textbf{homogéneas}, se caracterizan por la menor
  probabilidad de establecer vínculos entre posiciones sociales
  diferentes. Desde el punto de vista microsocial ello se acerca a una
  situación que tiende a la homofilia, donde la formación de relaciones
  sigue una pauta que revela correlación de status (y entre parámetros
  nominales o graduados), reduciendo el contacto entre grupos.
\item
  \textbf{Tamaño}: Las sociedades en las cuales se encuentran posiciones
  de diverso tamaño exhiben un incremento en la probabilidad de que
  miembros del grupo más pequeño interactúen con los integrantes del
  grupo mayoritario (Blau \& Schwartz, 1984).
\item
  Pequeños grupos privilegiados o excluidos pueden estar sujetos a
  normas de clausura o segregación que reducen el poder de las
  diferencias numéricas entre grupos; así por ejemplo ocurría con el
  ``apartheid'' sudafricano.
\end{itemize}
\end{frame}

\begin{frame}{Teoría IV}
\protect\hypertarget{teoruxeda-iv}{}
\begin{itemize}
\item
  \textbf{Interacciones micro}: Mark Granovetter (1973). De acuerdo con
  esta tesis, la clave de la cohesión social (integración, en la
  formulación original) reside en la capacidad para conectarse con
  círculos diversos antes que con un núcleo denso de personas cercanas
  (Granovetter, 1973).
\item
  La fuerza de los lazos se refiere a la frecuencia de contacto,
  confianza y compromiso emocional asociados con un vínculo
  (Granovetter, 1973).
\item
  Los \textbf{lazos fuertes} se asocian con similitud entre las partes
  en contacto, produciendo una estructura de pequeños grupos unidos
  hacia el interior por lazos fuertes y carentes de vínculos hacia otros
  círculos sociales.
\item
  La constatación de que los lazos fuertes reducen la probabilidad de
  éxito en procesos sociales relevantes como la difusión de información,
  la incorporación de innovaciones o la capacidad de acción colectiva.
  Esto lleva a Granovetter (1973) a sostener que son los \textbf{lazos
  débiles} los más relevantes para la cohesión social desde un punto de
  vista estructural.
\end{itemize}
\end{frame}

\begin{frame}{Teoría V}
\protect\hypertarget{teoruxeda-v}{}
\begin{itemize}
\item
  Las vinculaciones entre personas similares \textbf{cierran} el acceso
  a recursos, ideas, prácticas diferentes y actúan en contra de la
  integración del grupo (Espinoza, 1995). Esto último deviene en
  procesos acumulativos (y sinérgicos) de ventajas y desventajas de
  grupos (DiPrete \& Eirich, 2006; Jackson, 2021).
\item
  Diversos estudios hipotetizan sobre el \textbf{efecto sinérgico} de la
  homofilia cuando esta cruza diversos dominios de la vida social,
  entrecruzando personas similares en diferentes contextos (Degenne \&
  Forsé, 1999; Kalmijn \& Vermunt, 2007; Smith et al., 2014; Vermunt \&
  Kalmijn, 2006).
\item
  La hipótesis se alinea con el tercer teorema de Blau (1993) quien
  plantea que la homogeneidad social establece barreras a la interacción
  dada la correlación entre diferencias sociales en diversas
  dimensiones.
\end{itemize}
\end{frame}

\begin{frame}{Teoría VI (Homofilia y desigualdad)}
\protect\hypertarget{teoruxeda-vi-homofilia-y-desigualdad}{}
\begin{itemize}
\item
  La homofilia posee un papel clave en la separación, reproducción y
  evolución de grupos dentro de una estructura social y, en este
  sentido, en la configuración de desigualdades sociales (Jackson, 2021;
  Kossinets \& Watts, 2009).
\item
  \textbf{El papel de la homofilia redunda en ser un mecanismo que
  contribuye a la clausura grupal y a la acumulación de ventajas o
  desventajas sociales}
\item
  Dado que las personas dependen de sus redes para obtener información y
  oportunidades, las divisiones en una red generan información y
  oportunidades para permanecer concentradas en partes de la sociedad y
  no llegar a otras partes (Jackson, 2021; Rytina, 2020)
\item
  Es esperable que las dotaciones iniciales (por ejemplo, recursos
  financieros o culturales) pueden verse agravadas por influencia de la
  red y sus características, lo que exacerba la desigualdad entre grupos
  en la adopción de prácticas gratificantes en relación con lo que
  esperaríamos basándonos únicamente en las diferencias individuales
  (DiMaggio \& Garip, 2011, 2012; Rivera et al., 2010).
\end{itemize}
\end{frame}

\begin{frame}{Homofilia basal y endogámica}
\protect\hypertarget{homofilia-basal-y-endoguxe1mica}{}
\begin{itemize}
\item
  A nivel de redes personales, la literatura ha indicado dos maneras a
  través de las cuales la homofilia influye en la formación de diadas:
  1) En términos de \textbf{disponibilidad demográfica}, es decir, en
  forma de estructura de oportunidades para la interacción (Blau, 1977;
  Feld, 1981, 1982; McPherson et al., 2001) y, 2) en forma de
  \textbf{preferencias individuales por la similitud}.
\item
  La \textbf{homofilia basal} es entendida como el nivel de homofilia
  esperado de la mezcla aleatoria en la población y, la homofilia
  endogámica como el nivel de homofilia por encima de esa línea de base.
\item
  La \textbf{homofilia endogámica}, por lo tanto, incluye instancias que
  se clasificarían como homofilia de elección, pero también incluye
  cierta cantidad de homofilia inducida por que la homogeneidad de grupo
  puede ser un resultado de algún proceso de ``endogamia'', más allá de
  lo que está determinado por la distribución demográfica general
  (Kossinets \& Watts, 2009).
\end{itemize}
\end{frame}

\begin{frame}{Evidencia (Chile)}
\protect\hypertarget{evidencia-chile}{}
\begin{itemize}
\item
  Efectos negativos de la homogeneidad social en procesos de superación
  de la pobreza y movilidad social en general (Espinoza, 1995; Espinoza
  \& Canteros, 2001), y en procesos de coordinación de acción colectiva
  (Espinoza, 1999).
\item
  Diferencias regionales y culturales en la disponibilidad de lazos
  sociales (Espinoza \& Durston, 2013; Espinoza \& Rabí, 2009).
\item
  Otros estudios han vinculado directamente la sociabilidad en que prima
  la intimidad de la familia inmediata con los bajos niveles de
  confianza observables en la sociedad chilena (Torche \& Valenzuela,
  2011; Valenzuela \& Cousiño, 2000)
\item
  Homogamia en la clase media-alta: El carácter dominante de la
  homogamia se refleja en el tratamiento del matrimonio mixto como una
  transgresión de la norma; sólo en períodos de crisis social podría
  apreciarse mayor permisividad frente a los matrimonios de clase media
  con la aristocracia o élite económica (McClure, 2012).
\item
  La preferencia por la homogamia se expresa en las aprehensiones que
  muestran sus entrevistados con respecto de los matrimonios con
  personas que no pertenecen a la élite económica. La norma, sin
  embargo, puede relajarse ante las limitaciones demográficas y también
  económicas de los integrantes de la élite para cubrir todas las
  posiciones. El habitus hace que los futuros cónyuges lleguen a
  enamorarse de las personas que convienen a sus intereses.(Huneeuss,
  2013).
\item
  La homogamia educacional observada en parejas chilenas se ubica en
  niveles cercanos al 50\%, lo cual es semejante al nivel de Estados
  Unidos, pero menor al nivel de Brasil (Torche, 2010). La norma de
  homogamia en la clase alta encuentra validación en estos resultados,
  ya que el rasgo dominante en las parejas chilenas consiste en la menor
  propensión de quienes cuentan con educación universitaria a
  emparejarse con personas de menor educación.
\end{itemize}
\end{frame}

\begin{frame}{Evidencia (estudios de estratificación)}
\protect\hypertarget{evidencia-estudios-de-estratificaciuxf3n}{}
\begin{itemize}
\item
  Wright \& Choo (1992), estudian en qué medida las relaciones de
  explotación que sustentan las clases sociales resultan permeables a
  las relaciones de amistad. Los resultados muestran impermeabilidad en
  la dimensión de propiedad, mientras que las diferencias en las clases
  no propietarias, donde predominan relaciones de autoridad y
  calificación resultan más permeables a la amistad. La clase obrera, en
  particular, desarrolla relaciones de amistad con categorías
  superiores, como capataces, supervisores o autoempleados.
\item
  Chan \& Goldthorpe (2004) conceptualizan las relaciones de amistad a
  través de las clases como una jerarquía de status, la cual mapean
  sobre la tipología de clases EGP (Erikson \& Goldthorpe, 1992). Sus
  resultados muestran que la jerarquía de estatus tiende a ser más
  homogénea en algunas categorías de clase, notablemente en la élite de
  la clase de servicios, aunque están lejos de identificar una pauta.
\item
  Los anteriores resultados contrastan con la pauta reportada por
  Degenne (1999) para establecimientos industriales franceses en
  términos de mayores probabilidades de socialización en puntos
  intermedios de las distribuciones de clase o status. En este contexto,
  las posiciones extremas socializan `hacia adentro', con los
  oficinistas, técnicos y gerentes junior actuando como intermediarios.
\item
  La evolución de la homofilia en Francia muestra una reducción en
  términos de origen social u ocupación, aunque se mantiene
  relativamente estable en términos de niveles de educación y riqueza
  (Forsé \& Chauvel, 1995). La pauta converge con las réplicas del
  modelo de logro de status de Blau \& Duncan (1967) que muestran
  consistentemente el predominio de la educación del trabajador por
  sobre el status de su familia de origen.
\end{itemize}
\end{frame}

\begin{frame}{Relevancia}
\protect\hypertarget{relevancia-1}{}
\begin{itemize}
\item
  El enfoque de la homofilia como resultado de la composición de los
  grupos es una línea de trabajo escasamente desarrollada en América
  Latina; para Chile se cuenta solamente con un estudio sobre homogamia
  educacional (Torche, 2010).
\item
  \textbf{En suma, los estudios chilenos referidos a la homofilia
  tienden a concentrarse en el matrimonio -la homogamia- con énfasis en
  su estudio normativo. La formación de familias por cierto constituye
  un campo relevante para estudiar la homofilia en Chile, pero está
  lejos de constituir el único campo. Las relaciones de amistad, aún
  cuando algunos autores las asimilan a las relaciones familiares
  (Torche \& Valenzuela, 2011), siguen constituyendo un aspecto
  relevante de la sociabilidad chilena. Aún en el contexto de las
  mayores oportunidades de movilidad que ofrecen las sociedades
  modernas, la homofilia continúa siendo un rasgo preponderante}.
\end{itemize}
\end{frame}

\begin{frame}{Hipótesis por parámetro}
\protect\hypertarget{hipuxf3tesis-por-paruxe1metro}{}
\begin{itemize}
\item
  El \textbf{sexo} de las personas afecta sus posibilidades de formación
  de diádas. En específico, esperamos mayores niveles de homofilia entre
  mujeres. Estas pautas de roles estás asociadas con la norma de una
  sociedad patriarcal.
\item
  \textbf{Edad}, la etapa de \emph{transición demográfica} en que se
  encuentra chile muestra un peso creciente de los adultos mayores,
  quienes enfrentan condiciones de vida difíciles, por lo que se podría
  esperar menores niveles de cohesión social entre ellos. (Esta
  hipótesis no la podemos testear como edad como continúa)
\item
  \textbf{Educación}, (expación del sistema educativo especialmente
  desde el año 2000). El \emph{acceso a la educación} constituye un
  fuerte principio estructurador en la sociedad chilena, generalmente
  asociado con el status socioeconómico y, en sus niveles más altos, con
  la reproducción en los círculos sociales más deseables. En la medida
  que la fase de expansión en el acceso a la educación universitaria es
  relativamente reciente, puede esperarse mayor cohesión social entre
  las personas que hayan tenido mayor exposición al sistema formal de
  educación universitaria.
\item
  \textbf{Religión}, cae la fuerza de la fe católica como principio
  estructurante de vínculos. Esperamos ver mayores niveles de homofilia
  entre individuos evangélicos.
\item
  \textbf{Ideología política}, considerenado el incremento de la
  distancia entre partidos políticos y bases sociales, esperamos ver
  mayores niveles de polarización ideológica. Esto es, mayor
  probabilidad de formación de vinculos homófilos en los polos del
  espectro.
\item
  \textbf{Barrio}, esperamos bajas probabilidades de incluir a vecinos
  en las redes de confidentes.
\end{itemize}
\end{frame}

\begin{frame}{Metodología I}
\protect\hypertarget{metodologuxeda-i}{}
\begin{itemize}
\item
  Los datos de la red del ego ofrecen un medio para medir la prominencia
  de una o varias dimensiones demográficas.
\item
  Una dimensión de \textbf{baja prominencia}, o sin importancia,
  exhibirá bajos niveles de homofilia. En este caso los límites de
  categoriales son ``porosos'' y los individuos tienen menos
  restricciones para formar lazos sociales estrechos con miembros de un
  grupo (o categoría) social diferente.
\item
  Por su parte una categoría sociodemográfica \textbf{prominente}, o
  importante, en la organización de un sistema social destaca por
  establecer fuertes sesgos en la formación de vínculos estrechos entre
  individuos haciendo más probable la interacción entre individuos que
  pertenecen a esta categoría o grupo social.
\item
  En este sentido es relevante preguntarse si el nivel educativo
  alcanzado, es una división demográfica más importante que la religión,
  la edad o el género.
\item
  Si se recopila a lo largo del tiempo, es posible preguntarse cómo la
  prominencia de diferentes dimensiones aumenta / disminuye a la luz de
  cambios macroestructurales más grandes, como cambios demográficos (por
  ejemplo, el país es más diverso racialmente), aumento de la
  desigualdad y cambios en la segregación residencial (Smith et al.,
  2014; Smith \& Gauthier, 2020).
\end{itemize}
\end{frame}

\begin{frame}{Metodología II}
\protect\hypertarget{metodologuxeda-ii}{}
\begin{itemize}
\item
  En este trabajo seguimos la orientación de Smith et al.~(2014):
  revisamos la \textbf{tasa absoluta} de interacción o contacto entre
  grupos sociodemográficos, así como la \textbf{tasa de contacto en
  relación con algún modelo de base de expectativas de azar}.
\item
  Las expectativas al azar se construyen en función de la composición
  demográfica de la población de interés.
\item
  Esta preocupación está fundamentada en la importancia del aprendizaje
  que podemos desarrollar mirando ``todos los mundos posibles'' (Watts,
  2003), y la comprensión de la importancia relativa del orden y la
  aleatoriedad como parámetros que podemos sintonizar para movernos a
  través de un espacio de posibilidades.
\item
  Se pueden emplear varios modelos para evaluar la prominencia relativa
  al azar: incluidos los \textbf{modelos log-lineales} tradicionales,
  \textbf{modelos de casos y controles}, \textbf{modelos de gráficos
  aleatorios exponenciales} y \textbf{modelos de espacios latentes}. En
  este trabajo usamos la aproximación de las regresiones caso-control. A
  continuación describimos en breve este procedimiento.
\end{itemize}
\end{frame}

\begin{frame}{Metodología III (Regresiones logísticas caso-control)}
\protect\hypertarget{metodologuxeda-iii-regresiones-loguxedsticas-caso-control}{}
\begin{itemize}
\item
  Los modelos de casos y controles se utilizan a menudo en la
  investigación médica para estudiar enfermedades raras que son
  difíciles de capturar mediante un muestreo aleatorio (Borgan et al.,
  2018).
\item
  Los modelos comparan los casos, un conjunto de individuos con la
  ``enfermedad'', con los controles, un conjunto de individuos sin la
  ``enfermedad''.
\item
  Un modelo de casos y controles es ideal para los datos de la red del
  ego porque la muestra captura el raro evento de interés= \textbf{las
  relaciones sociales entre actores} (Smith et al., 2014).
\item
  Todos los \textbf{casos} son parejas Ego-alteri, las cuales mantienen
  una relación social conocida (en nuestro caso, de confidentes).
\item
  Los \textbf{controles} representan una muestra aleatoria de parejas
  que no tienen una relación social conocida. Esto se forma emparejando
  aleatoriamente a los encuestados de la muestra, capturando la mezcla
  aleatoria en la población. El emparejamiento aleatorio se realizda
  considerando pesos muestrales.
\end{itemize}
\end{frame}

\begin{frame}{Metodología IV}
\protect\hypertarget{metodologuxeda-iv}{}
\begin{itemize}
\item
  Para las variables numéricas, como la edad, esto se mide como la
  diferencia absoluta entre i y j.
\item
  Para variables categóricas, como raza o religión, la distancia se mide
  como un término coincidente (¿i y j tienen el mismo nivel educativo?).
\item
  \textbf{En resumen, la distancia sociodemográfica entre los
  encuestados y los alters, o los casos, se compara con la distancia
  sociodemográfica entre los encuestados emparejados aleatoriamente o
  los controles. Por lo tanto, el modelo compara la distancia
  sociodemográfica observada en los datos con la esperada bajo una
  mezcla aleatoria en la población (Smith et al., 2014)}.
\end{itemize}
\end{frame}

\begin{frame}{Modelo}
\protect\hypertarget{modelo}{}
\[
ln\frac{\ p(Y)}{1- p(Y)} = \theta D \qquad (1)
\]

donde \(Y_{ij}\) es la presencia o ausencia de un empate, \(D_{ij}\) es
la distancia sociodemográfica entre i y j para cada díada, y \(\theta\)
es el vector de coeficientes.

\begin{itemize}
\item
  El modelo de control de casos es útil debido a que permiten evaluar la
  prominencia relativa de los parámetros, es decir, controlando por los
  efectos de otros parámetros.
\item
  Esto es importante, puesto que la homofilia suele operar como una
  tendencia sinérgica, es decir, como fuerza de atracción simultánea a
  través de distintos parámetros.
\item
  Adicionalmente, su interpretación es más directa y sencilla que la que
  debemos realizar si utilizamos modelos loglineales, en donde podemos
  evaluar la prominencia entre máximo dos parámetros a la vez.
\end{itemize}
\end{frame}

\begin{frame}{Data}
\protect\hypertarget{data}{}
\begin{itemize}
\tightlist
\item
  Los datos utilizados corresponden con ENACOES (2014), ELSOC (2017 y
  2019).
\end{itemize}
\end{frame}

\begin{frame}{Resultados I}
\protect\hypertarget{resultados-i}{}
\end{frame}

\begin{frame}{Resultados II (Regresiones)}
\protect\hypertarget{resultados-ii-regresiones}{}
\end{frame}

\begin{frame}{Resultados III}
\protect\hypertarget{resultados-iii}{}
\end{frame}

\begin{frame}{Resumen resultados}
\protect\hypertarget{resumen-resultados}{}
En general, los resultados son consistentes con lo evidenciado por
(Bargsted et al., 2020) para el año 2014.

\begin{itemize}
\item
  \textbf{Edad}, no tiene resultados significativos excepto para el año
  2014. Este resultado indica mayor probabilidad de realcionarse entre
  personas con edades similares, especificamente, en entornos no
  familiares.
\item
  \textbf{Sexo}, los resultados muestran consistentemente más homofilia
  entre mujeres, y en entornos no familiares.
\item
  \textbf{Nivel educativo}, los resultados sugieren que el ayor nivel de
  homofilia se da entre individuos con nivel educativo Universitario.
  Esto ocurre de manera muy similar considerando todos los vinculos y
  vinculos familiares. Si consideramos los años 2017-2019, observamos un
  decaimiento en la fuerza de los efectos, y una mayor concentración de
  homofilia en el nivel educativo menos que media.
\item
  \textbf{Religión}, los resultados muestran consistentemente mayores
  niveles de homofilia para los evangélicos y menores niveles de
  homofilia en entornos no familiares.
\item
  \textbf{Ideología política}, los mayores niveles de homofilia se
  encuentran en los extremos del espectro ideológico, en particular
  entre aquellos de ideología de extrema derecha. Los que se identifican
  con una posición ideológica de centro son los menos homofílicos.
\item
  \textbf{Barrio}, los resultados muestran consistentemente muy bajas
  probabilidades de incluir a vecinos en las redes personales de
  confidentes.
\end{itemize}
\end{frame}

\begin{frame}{Discusión}
\protect\hypertarget{discusiuxf3n}{}
\begin{itemize}
\item
  El mayor nivel de homofilia entre mujeres puede estar relacionado a
  las menores tasas de integración en el mercado laboral en comparación
  con los hombres. Esto puede ser considerado como una dimensión
  relevante de las desigualdad de genero en tanto denota pautas de
  ventaja en el acceso a mayor capital social por parte de los hombres.
\item
  La ideología y el nivel educativo parecen ser los parámetros más
  prominentes. En estas dos dimensiones los polos se muestran altamente
  clausurados. Esto sugiere una correlación entre parámetros y una
  dimensión relevante de la fragmentación vertical de lo social y lo
  político.
\item
  Pautas de segregación interseccionales: El credo religioso mantiene su
  relvancia, especialmente en entornos evangélicos. En tanto que la
  mayor presencia de este tipo de credo se da en los sectores populares,
  puede ser considerado como un parámetro adicional de clausura e
  integración local en los sectores más desfavorecidos de la población.
\end{itemize}
\end{frame}

\end{document}
